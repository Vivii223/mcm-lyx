% https://www.overleaf.com/learn/latex/LaTeX_Graphics_using_TikZ:_A_Tutorial_for_Beginners_(Part_3)%E2%80%94Creating_Flowcharts

\documentclass{article}

\usepackage{tikz}
\usetikzlibrary{shapes.geometric, arrows}
\usepackage{graphicx}

\begin{document}

% start & stop blocks,矩形
\tikzstyle{startstop} = [rectangle, rounded corners, minimum width=3cm, minimum height=1cm,text centered, draw=black, fill=red!30]
% input & output box,梯形
\tikzstyle{io} = [trapezium, trapezium left angle=70, trapezium right angle=110, minimum width=3cm, minimum height=1cm, text centered, text width=3cm, draw=black, fill=blue!30]
% process blocks,矩形
% decision blocks,菱形
\tikzstyle{process} = [rectangle, minimum width=3cm, minimum height=1cm, text centered, text width=3cm, draw=black, fill=orange!30]
\tikzstyle{decision} = [diamond, minimum width=3cm, minimum height=1cm, text centered, text width=3cm, draw=black, fill=green!30]
% style for the arrows
\tikzstyle{arrow} = [thick,->,>=stealth]

\scalebox{0.7}{
\begin{tikzpicture}[node distance=2cm]

% \node (label) [style name] {text}
\node (start) [startstop] {Start};
\node (in1) [io, below of=start] {Initialize};
\node (pro1) [process, below of=in1] {For each time step};
\node (pro2) [process, below of=pro1] {For each vehicle};
\node (pro3) [process, below of=pro2] {Determine speed and location};
% \node (dec1) [decision, below of=pro1] {Decision 1};
%  Rotation, see https://tex.stackexchange.com/questions/16870/decision-node-on-tikz-diamond
\node (dec1) [decision, below of=pro3, yshift=-0.5cm, aspect=3] {Can it change to the left lane?};
\node (dec2) [decision, right of=dec1, xshift=5cm, aspect=3] {Can it change to the right lane?};
\node (pro4) [process, below of=dec1] {Change to the left lane};
\node (pro5) [process, below of=dec2] {Change to the right lane};

% Arrows
\draw [arrow] (start) -- (in1);
\draw [arrow] (in1) -- (pro1);
\draw [arrow] (pro1) -- (pro2);
\draw [arrow] (pro2) -- (pro3);
\draw [arrow] (pro3) -- (dec1);
\draw [arrow] (dec1) -- node[anchor=east] {yes} (pro4);
\draw [arrow] (dec1) -- node[anchor=south] {no} (dec2);
\draw [arrow] (dec2) -- node[anchor=east] {yes} (pro5);

\node (dec3) [decision, above of=dec2, yshift=1cm, aspect=3] {Have we processed all the vehicles?};
\draw [arrow] (dec2) -- node[anchor=east] {no} (dec3);
\draw [arrow] (pro4) -- ++(0:3.2cm) -- ++(90:3.7cm) -- ++(0:3.8cm);

\node (pro6) [process, above of=dec3, yshift=-0.5cm, xshift=-3cm] {Process next vehicle};
\draw [arrow] (dec3.north) -- ++(90:0.5cm) -- node[anchor=north] {no} (pro6);
\draw [arrow] (pro6) -- (pro2);
\node (dec4) [decision, right of=dec3, yshift=3.5cm, xshift=3cm, aspect=3] {Have the time step reached the pre-determined value?};
\draw [arrow] (dec3.east) -- ++(0:2.05cm) -- node[anchor=east] {yes} (dec4.south);
\draw [arrow] (dec4.west) -- node[anchor=south] {yes} (pro1.east);

\node (stop) [startstop, above of=dec4, yshift=3cm] {Stop};
\draw [arrow] (dec4.north) -- node[anchor=east] {no} (stop.south);

\end{tikzpicture}}
\end{document}